\section{Diagonalization}


There are two questions for a linear operator $T$:
\begin{enumerate}
    \item Is there an ordered basis $\beta$ that $[T]_\beta$ is a diagonal matrix?
    \item If such basis exists, how can it be found?
\end{enumerate}


\subsection{Eigenvalue and Eigenvectors}

\begin{definition}
    A linear operator $T$ on $V$ is \cindex{diagonalizable} if there is an ordered basis $\beta$ of $V$ that $[T]_\beta$ is a diagonal matrix. A matrix is \cindex{diagonalizable} if $L_A$ is diagonalizable.


If an operator $T$ is diagonalizable, for $\beta = \{v_i\}$, we have
\begin{equation*}
    T(v_j) = \sum_{i=1}^n D_{ij} v_j = D_{jj} v_j = \lambda_j v_j
\end{equation*}
So to prove a  linear operator $T$ is diagnolizable is to find a basis $\beta = \{ v_i \}$ and $\{ \lambda_j \}$ that $T(v_i) = \lambda_i v_i$.
\qed
\end{definition}

\begin{definition}
    A \emph{non-zero} vector $v \in V$ is called an \cindex{eigenvector} of linear operator $T$ if $\exists \lambda : T(v) = \lambda v$. $\lambda$ is called \cindex{eigenvalue} corresponding to eigenvector $v$. So it is for matrix. A eigenvalue could be $0$, but eigenvector could not be $\vec{0}$.
\end{definition}

Eigenvector is also called \cindex{characteristic vector}. Eigenvalue is also called \cindex{characteristic value}.

\begin{theorem}
    A linear operator $T$ is diagonalizable if there exists an ordered basis consisting of eigenvectors of $T$.
\end{theorem}

\begin{theorem}
    $\lambda$ is an eigenvalue of $A$ $\iff$ $\determinate{A - \lambda I_n} = 0$.
\end{theorem}

\begin{proof}
    If $\lambda$ is an eigenvalue of $A$, $\exists v \in F^n, v \neq 0$ that $A v = \lambda v$, which is $(A - \lambda I_n)(v)= 0$, which means $A - \lambda I_n$ is not invertible because $v \neq 0$, so $\determinate{A - \lambda I_n} = 0$.
\end{proof}

\begin{theorem}
    Every eigenvalue has at least one eigenvector.    
\end{theorem}
\begin{proof}
    Since $\determinate{A - \lambda I_n} = 0$, $(A - \lambda I_n) x = 0$ is a homogeneous equation with $\dimension{A - \lambda I_n} < n$.
\end{proof}



\begin{definition}
    For $A = [T]_\beta$ the polynomial $f(t) = \determinate{A - t I_n}$ is called the \cindex{characteristic polynomial} of $A$ and $T$. 
\end{definition}

\begin{theorem}
    The choice of basis $\beta$ did not change the eigenvalue of $T$. 
\end{theorem}
\begin{proof}
    \begin{equation*}
        \absolutevalue{[T]_\beta - \lambda I} = \absolutevalue{Q^{-1} \left( [T]_\alpha - \lambda I \right) Q} = \absolutevalue{Q^{-1}} \times \absolutevalue{[T]_\alpha - \lambda I } \times \absolutevalue{Q} = \absolutevalue{[T]_\alpha - \lambda I } 
    \end{equation*}
\end{proof}

\begin{theorem}
Similar matrices have the same characteristic function.    
\end{theorem}



\begin{theorem}
    $v$ is an eigenvector of $T$ corresponding to $\lambda$ $\iff$ $v \neq 0 \wedge v \in \nullspace{T - \lambda I}$.
\end{theorem}

\begin{theorem}
    if $Q$ is a matrix with columns of eigenvectors of $\beta$, then according to theorem (\ref{specialchangeofcoordinates}), $Q^{-1} A Q$ is a diagonal matrix with eigenvalue.
\end{theorem}



\subsection{Diagonalizability}


\begin{theorem}
    Let ${\lambda_i}$ be distinct eigenvalue of $T$. If $\{v_i\}$ are eigenvector that corresponding to $\lambda_i$, then $\{v_i\}$ is \emph{linearly independent}.
\end{theorem}
\begin{proof}
    suppose it works for $k - 1 \geq 1$ and we have $k$ eigenvector $\{ v_i\}$. Suppose
    \begin{equation*}
        a_1 v_1 + a_2 v_2 + \dots + a_k v_k = 0
    \end{equation*}
    
    multiply $T - \lambda_k I$ to both sides, we have
    \begin{equation*}
        a_1(\lambda_1 - \lambda_k) v_1 + a_1(\lambda_2 - \lambda_k) v_2 +  \dots + a_1(\lambda_{k-1} - \lambda_k) v_{k-1} +  = 0
    \end{equation*}
    
    because $\{v_1, v_2, \dots, v_{k-1} \}$ are linearly independent, we have 
    \begin{equation*}
        a_1(\lambda_1 - \lambda_k) = a_1(\lambda_2 - \lambda_k) =  a_1(\lambda_{k-1} - \lambda_k) = 0
    \end{equation*}
    
    because $\lambda_i$ are different, we have $a_i = 0$.
\end{proof}

\begin{theorem}
    if $T$ has $n$ distinct eigenvalues, then $T$ is diagonalizable. If $T$ is diagonalizable, it may not have $n$ distinct eigenvalues, for example the identity matrix $I_V$.
\end{theorem}

\begin{definition}
    A polynomial $f(t)$ in $P(F)$ \cindex{split over} $F$ if there are scalars $c, a_1, \dots, a_n$ (not necessarily distinct) in $F$ that
    \begin{equation*}
        f(t) = c(t - a_1)(t - a_2) \dots (t-a_n)
    \end{equation*}
\end{definition}

\begin{theorem}
    the characteristic polynomial of any diagonalizable linear operator splits.
\end{theorem}

\begin{proof}
    choose a basis $\beta$ of eigenvectors. $[\mathrm{T}]_\beta$ is a diagonal matrix $D$. The characteristic polynomial of $T$ is $|D - tI|$ splits.
\end{proof}

Be careful that the characteristic polynomial splits does not mean the matrix is diagonalizable. The eigenvectors need to form a basis.

\begin{definition}
    the \cindex{multiplicity} of $\lambda$ is the largest positive integer $k$ for which $(t - \lambda)^k$ is a factor of $f(t)$.
\end{definition}

\begin{definition}
    let $\lambda$ be an eigenvalue of $T$. Let $E_\lambda = \nullspace{T - \lambda I_V}$. the set $E_\lambda$ is called the \cindex{eigenspace} of $T$ corresponding to eigenvalue $\lambda$. So is it for matrix.
\end{definition}

\begin{theorem}
    let $\lambda$ be an eigenvalue of $T$ having multiplicity $m$. then $1 \leq \dimension{E_\lambda} \leq m$.
\end{theorem}
\begin{proof}
    choose ordered basis $\{v_1, v_2, \dots, v_p\}$ for $E_\lambda$, and extend it to ordered basis $\beta = \{ v_1, v_2, \dots, v_p, v_{p+1}, \dots, v_n$ for $V$, and let $A = [T]_\beta$. let $v_i (1 \leq i \leq q)$ be an eigenvector of $T$ corresponding to $\lambda$, we have
    \begin{equation*}
        A = \begin{pmatrix}
            \lambda I_p & B \\
            0 & C
        \end{pmatrix}
    \end{equation*}
    so \begin{equation*}
        \begin{aligned}
            f(t) &= \absolutevalue{A - t I_n} \\
            &= \absolutevalue{\begin{pmatrix}
                (\lambda - t) I_p & B \\
                0 & C - t I_{n-p}
            \end{pmatrix}} \\
            &= \absolutevalue{(\lambda - t)I_p} \times \absolutevalue{C - t I_{n-p}} \\
            &= (\lambda - t)^p g(t)
        \end{aligned}
        \end{equation*}
    So $(\lambda - t)^p$ is a factor of $f(t)$, and the multiplicity of $\lambda$ is at least $p = \text{dim}(E_\lambda)$, so $\text{dim}(E_\lambda) \leq m$ 
\end{proof}

\begin{theorem}
    let $\lambda_1, \lambda_2, \dots, \lambda_k$ be distinct eigenvalue of $T$. let $S_i$ be a finite linearly independent subset of eigenspace $E_{\lambda_i}$. then $S_1 \cup S_2 \cup \dots \cup S_k$ is a linearly independent subset of $V$.
\end{theorem}

\begin{theorem}
    let $\lambda_1, \lambda_2, \dots, \lambda_k$ be distinct eigenvalue of $T$, then
    \begin{enumerate}
        \item $T$ is diagonalizable $\iff$ the multiplicity of $\lambda_i$ is equal to $\dimension{E_{\lambda_i}}$ for all $i$.
        \item If $T$ is diagonalizable and $\beta_i$ is an ordered basis for $E_{\lambda_i}$ for each $i$, then $\beta = \beta_1 \cup \beta_2 \cup \dots \cup \beta_k$ is an ordered basis for $V$ consisting of eigenvectors of $T$.
    \end{enumerate}
\end{theorem}

\begin{theorem}
    $T$ is diagonalizable $\iff$ both of the following holds:
    \begin{enumerate}
        \item the characteristic polynomial of $T$ splits.
        \item for each eigenvalue $\lambda$ of $T$, the multiplicity of $\lambda$ equals $n - \rank{T - \lambda I}$.
    \end{enumerate}
\end{theorem}

\begin{definition}
    Let $W_i$ be subspaces of a vector space $V$. The \cindex{sum} of these subspaces is defined as:
    \begin{equation}
        \sum_{i=1}^k W_i = \{ v_1 + v_2 + \dots + v_k : v_i \in W_i \text{ for } 1 \leq i \leq k \}
    \end{equation}
\end{definition}

\begin{definition}
    let $W_i$ be subspace of $V$. $V$ is the \cindex{direct sum} of subspace $W_1, W_2, \dots, W_k$, or $V = W_1 \oplus W_2 \oplus \dots \oplus W_k$ if
    \begin{equation*}
        V = \sum_{i=1}^k W_i
    \end{equation*}
    and 
    \begin{equation*}
        W_j \cap \sum_{i \neq j} W_i = \emptyset, (1 \leq j \leq k)
    \end{equation*}
\end{definition}

\begin{theorem}
    $T$ is diagonalizable $\iff$ $V$ is the direct sum of eigenspaces of $T$.
\end{theorem}


\subsection{Invariant Subspaces}

\begin{definition}
    A subspace $W$ of $V$ is $T$-\cindex{invariant subspace} of $V$ if $T(W) \subseteq W$.
    Common $T$-invariant subspaces are: $\emptyset$, $V$, $R(T)$, $N(T)$.
    \qed
\end{definition}



\begin{definition}
    A $T$-\cindex{cyclic subspace} of $V$ generated by $x$ is defined as $W=\text{span} \left( \left \{x, T(x), T^2(x), \dots \right \}\right)$.
\end{definition}

\begin{theorem}
    Let $T$ be a linear operator on finite-dimensional vector space $V$, and let $W$ be a $T$-invariant subspace of $V$. Then the characteristic polynomial of $T_W$ divides the characteristic polynomial of $T$.
\end{theorem}

\begin{proof}
    Choose ordered basis $\gamma$ for $W$ and expand it to $\beta$ for $V$. Calculate $[T]_\beta$ and $[T]_\gamma$.
\end{proof}


\begin{theorem}
    Let $T$ be a linear operator on finiate-dimensional vector space $V$, and let $W$ be a $T$-cyclic subspace of $V$ generated by nonzero vector $v \in V$. Let $k = \dimension{W}$. Then:
    \begin{enumerate}
        \item $\left\{ v, T(v), T^2(v), \dots, T^{k-1}(v) \right\}$ is a basis for $W$.
        \item If $a_0 v + a_1 T(v) + a_2 T^2(v) + \dots + a_{k-1} T^{k-1}(v) + T^k(v) = 0$, then the characteristic polynomial of $T_W$ is $f(t) = (-1)^k (a_0 + a_1 t + \dots + a_{k-1} t^{k-1} + t^k)$.
    \end{enumerate}
\end{theorem}

\begin{proof}
    Let $\beta = \left\{v, T(v), T^2(v), \dots, T^{k-1}(v) \right\}$, and let $a_i$ be the scalars that 
    \begin{equation*}
            a_0 v + a_1 T(v) + a_2 T^2(v) + \dots + a_{k-1} T^{k-1}(v) + T^k(v) = 0
    \end{equation*}
    
    For each basis $\left\{v, T(v), T^2(v), \dots, T^{k-1}(v) \right\}$, $[T(v)]_\beta = [0,1,\dots]$, $T(T(v))_\beta = [0,0,1,\dots]$, etc, we have:
    \begin{equation*}
        [T_W]_\beta = \begin{pmatrix}
            0 & 0 & \dots & 0 & - a_0 \\
            1 & 0 & \dots & 0 & -a_1\\
            \vdots & \vdots  & & \vdots & \vdots \\
            0 & 0 & \dots & 1 & - a_{k-1}
        \end{pmatrix}
    \end{equation*}
    which has characteristic polynomial 
    \begin{equation*}
        f(t) = (-1)^k (a_0 + a_1 t + \dots + a_{k-1} t^{k-1} + t^k)
    \end{equation*}
\end{proof}

\begin{theorem}[\cindex{Cayley-Hamilton}]
    Let $T$ be linear operator on a finite-dimensional vector space $V$, and let $f(t)$ be the characteristic polynomial of $T$. Then $f(T) = 0$.
\end{theorem}

\begin{proof}
    Suppose $v \neq 0$. Let $W$ be the $T$-cyclic subspace generated by $v$, and suppose the $\text{dim}(W)=k$. So there exists scalars $\{a_i\}$ that 
    \begin{equation*}
        a_0 v + a_1 T(v) + a_2 T^2(v) + \dots + a_{k-1} T^{k-1}(v) + T^k(v) = 0
    \end{equation*}
    which implies the characteristic polynomial of $T_W$ is
    \begin{equation*}
        g(t) = (-1)^k (a_0 + a_1 t + \dots + a_{k-1} t^{k-1} + t^k)
    \end{equation*}
    We have
    \begin{equation*}
        g(T)(v) = (-1)^k (a_0 I + a_1 T + \dots + a_{k-1} T^{k-1} + T^k)(v) = 0
    \end{equation*}
    Because $g(t)$ divides $f(t)$, $\exists q(t)$ that $f(t) = g(t) q(t)$. So
    \begin{equation*}
        f(T)(v) = q(T)g(T)(v) = q(T) (g(T)(v)) = q(T)(0) = 0
    \end{equation*}
\end{proof}



\begin{definition}
    Let $B_1 \in M_{m \times m}(F)$, and $B_2 \in M_{n \times n} (F)$. The \cindex{direct sum} of $B_1$ and $B_2$, denoted as $B_1 \oplus B_2$, as the $(m+n) \times (m+n)$ matrix $A$ that
    \begin{equation*}
        A = \begin{pmatrix}
            B_1 & 0 \\
            0 & B_2
        \end{pmatrix}
    \end{equation*}
\end{definition}


\begin{theorem}
    Suppose $V=W_1 \oplus W_2 \oplus \dots \oplus W_k$, where $W_i$ is a $T$-invariant subspace of $V$. Suppose $f_i(t)$ is the characteristic polynomial of $T_{W_i}$, Then $f_1(t) f_2(t) \dots f_k(t)$ is the characteristic polynomial of $T$. Let $\beta_i$ be an ordered basis for $W_i$, and let $\displaystyle \beta = \bigcup_{i=1}^k \beta_i$. Let $A=[T]_\beta$, and $B_i=[T_{W_i}]_\beta$. Then $A = B_1 \oplus B_2 \oplus \dots \oplus B_k$.
\end{theorem}