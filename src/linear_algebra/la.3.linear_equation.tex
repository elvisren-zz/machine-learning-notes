\section{Linear Equations}

\subsection{Elementary Operations}

\begin{definition}
	Let $A$ be an $m\times n$ matrix. there are three \cindex{elementary row operation}:
	\begin{enumerate}
		\item interchange any two row of $A$.
		\item multiply any row of $A$ by nonzero scalar.
		\item add any scalar multiple of a row of $A$ to another row.
	\end{enumerate}
\end{definition}

\begin{definition}
	An $n\times n$ \cindex{elementary matrix} is a matrix obtained by performing \emph{one} elementary operation on $I_n$.
\end{definition}

\begin{definition}
	The \cindex{rank} of $A_{m \times n}$, denoted $\text{rank}(A)$, is the rank\footnote{The rank of a linear transformation is defined in Definition (\ref{rankdefinition}) on page \pageref{rankdefinition}.} of linear transformation $L_A: F^n \rightarrow F^m$.
\end{definition}

\begin{theorem}
	the rank of a matrix equals the maximum number of linearly independent columns.
\end{theorem}
\begin{proof}
	For any $A \in M_{m\times n}(F)$, 
	\begin{equation*}
		\begin{aligned}
			\text{rank}(A) &= \text{rank}(L_A) = \text{dim}(R(L_A)) = \text{span}(L_A(\beta)) \\
			&= \text{span}(\{ L_A(e_1), L_A(e_2), \dots, L_A(e_n) \})
		\end{aligned}
	\end{equation*}
	we have $L_A(e_j) = A e_j = a_j$ where $a_j$ is the $j$th column of A. Hence
	\begin{equation*}
		R(L_A) = \text{span}(\{ a_1, a_2, \dots, a_n \})
	\end{equation*}
\end{proof}

\begin{theorem}
    Let $A_{m \times n}$ has rank $r$. Then there exist invertible matrix $B_{m \times m}$ and $C_{n \times n}$ that $D=BAC$, where:
    \begin{equation*}
        D = \begin{pmatrix}
			I_r & 0 \\
			0 & 0 \\
		\end{pmatrix}
    \end{equation*}
\end{theorem}


\begin{theorem}
    Every invertible matrix is a product of elementary matrices.
\end{theorem}

\begin{definition}
	For system $Ax=b$, the matrix $(A|b)$ is the \cindex{augmented matrix}.
\end{definition}


\begin{theorem}
    If A is an invertible matrix, it is possible to transform augmented matrix $(A|I_n)$ into matrix $(I_n|A^{-1})$ by means of a finite number of elementary row operations.
\end{theorem}

\subsection{System of Equations}

\begin{definition}
	A system $A_{m \times n} x=b$ of $m$ linear equation in $n$ unknowns is \cindex{homogeneous} if $b=0$. Otherwise the system is \cindex{nonhomogeneous}.
\end{definition}

\begin{definition}
	A system is \cindex{consistent} if its solution set is not empty. otherwise it is called \cindex{inconsistent}.
\end{definition}

\begin{theorem}
	Let $K$ be the set of all solutions for $Ax=0$. Then $K=\nullspace{L_A}$ has dimension of $n- \rank{L_A}=n-\rank{A}$.
\end{theorem}

\begin{theorem}
	if $m < n$, the system $Ax=0$ has nonzero solution.
\end{theorem}
\begin{proof}
    $\rank{A} \leq m < n$, so $\nullspace{A} = n - \rank{A} > 0$.
\end{proof}

\begin{theorem}\label{equationfromoneandnullspace}
	Let $K$ be the solution set of $Ax=b$, $K_H$ be the solution set of $Ax=0$. Then for all solution $s$ to $Ax=b$,
	\begin{equation}
		K = \{ s \} + K_H = \{s+k: k \in K_H \}
	\end{equation}
\end{theorem}


\begin{theorem}
	Let $A_{n \times n}x=b$ be a system of equations. If $A$ is invertible, the solution is $A^{-1}b$. Conversely, if the system has exactly one solution, $A$ is invertible.
\end{theorem}



\begin{theorem}
	Let $Ax=b$ be a system. the system is consistent $\Leftrightarrow$ $\text{rank}(A) = \text{rank}(A|b)$.
\end{theorem}

\begin{proof}
    $R(L_A) = \text{span}(\{a_1, a_2, \dots, a_n \})$. Since $b \in R(L_A)$, the extended span is the same.
\end{proof}


\begin{definition}
	A matrix is in \cindex{reduced row echelon form} if 
	\begin{enumerate}
		\item any row containing a nonzero entry precedes any row in which all the entries are zero.
		\item the first nonzero entry in each row is the only nonzero entry in its column.
		\item the first nonzero entry in each row is $1$ and it occurs in a column to the right of the first nonzero entry in the preceding row.
	\end{enumerate}
\end{definition}

\begin{theorem}\label{rankoftwomatrix}
    For $A_{m \times n}$ and $B_{n \times p}$, we have:
    \begin{equation}
        \rank{AB} = \rank{B} - \dimension{\nullspace{A} \cap \rangespace{B}}
    \end{equation}
\end{theorem}
\begin{proof}
    Let $\beta_i$ be the basis of $\nullspace{A} \cap \rangespace{B}$, expand to the basis $\beta \cup \alpha$ of $B$. Prove $\alpha$ is a basis of $\rangespace{AB}$.
\end{proof}

\begin{theorem}\label{rankofadjoint}
    For $A_{m \times n}$, we have
    \begin{enumerate}
        \item $\rank{A^\top A} = \rank{A} = \rank{A A^\top}$.
        \item $\rangespace{A^\top A} = \rangespace{A^\top}$.
        \item $\nullspace{A^\top A} = \nullspace{A}$.
    \end{enumerate}
    $A^\top$ could be replaced by $A^*$ in $C$.
\end{theorem}
\begin{proof}
    If $\exists x \neq 0 \left(x \in \nullspace{A^\top} \cap \rangespace{A} \right)$. Then $(A^\top x = 0) \wedge \left(\exists y(x = A y) \right)$. So $x^\top x = y^\top A^\top x = y^\top ( A^\top x) = 0 $ and then $x =0$. According to \thmref{rankoftwomatrix}, $\rank{A^\top A} = \rank{A^\top} - \dimension{\nullspace{A^\top} \cap \rangespace{A}} = \rank{A}$.
\end{proof}

\begin{theorem}
    For a system of linear equation $Ax = b$, the associated system of \cindex{normal equations} is defined as $n \times n$ system
    \begin{equation}
        A^\top A x = A^\top b
    \end{equation}
    
    $A^\top A x = A^\top b$ is always consistent and has unique solution when $\rank{A} = n$. If $Ax=b$ is consistent, two solutions are the same.
\end{theorem}



